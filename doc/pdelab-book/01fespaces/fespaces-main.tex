\section{General construction of finite element spaces}

Linear finite element spaces are constructed in a generic and
systematic way. It therefore desirable to implement as much as
possible of this construction in a reusable form. In order to identify
which pieces can be reused we now describe the general construction in
detail. 

We begin with the shape functions which we also call
local basis functions. These are functions defined
on a reference element $\hat\Omega$:
\begin{equation}
\hat\phi_i : \hat\Omega \to \mathbb{K}^m, \qquad 0\leq i < N.
\end{equation}
Here $\mathbb{K}^m$ is the $m$-dimensional vector space based on the
field $\mathbb{K}$ where $\mathbb{K}=\mathbb{R}, \mathbb{C}$.
Typically, shape functions are componentwise polynomials of a given
order. 
A local finite element function 
\begin{equation*}
\hat{u}(x) = \sum_{i=0}^{N-1} \hat{c}_i \hat\phi_i(x) .
\end{equation*}
is then a linear combination of those
local basis functions.

To define a function on a general transformed element $e$ we introduce
the transformation 
\begin{equation}
\mu_e : \hat\Omega_e \to \Omega_e, \qquad e\in E^0
\end{equation}
which is assumed to be continuous and invertible.
Here $\hat\Omega_e$ is the reference element associated with element
$e$ and $\Omega_e$ is the domain occupied by element $e$. Note also
that every element may in principle have a different reference element.

Using the transformation we can define the transformed shape functions
\begin{equation}
\phi_i^e : \Omega_e \to \mathbb{K}^m, \qquad \phi_i^e =
\hat\phi_i^e\circ \mu_e^{-1}, \qquad 0\leq i < N_e.
\end{equation}
and a function on the transformed element
\begin{equation}
u_e(x) = \sum_{i=0}^{N_e-1} \hat{c}_i^e \hat\phi_i^e(\mu_e^{-1}(x)) .
\end{equation}

The domain $\Omega$ is assumed to be subdivided into elements
\begin{equation}
\bar{\Omega} = \bigcup_{e\in E^0} \bar\Omega_e, \qquad
\Omega_e\cap\Omega_f = \emptyset \text{ for $e\neq f$}.
\end{equation}
A global finite element function $$u : \Omega \to \mathbb{K}^m$$ is then
constructed out of the local pieces:
\begin{equation}
u(x) = \sum_{e\in E^0} \left( \sum_{i=0}^{N_e-1} \hat{c}_i^e
\hat\phi_i^e(\mu_e^{-1}(x)) \right) \chi_{\Omega_e}(x), 
\qquad x\in \bigcup_{e\in E^0} \Omega_e
\end{equation}
where $\chi_{\Omega_e}$ denotes the characteristic function of the
domain $\Omega_e$. Note that:
\begin{itemize}
\item Each element still has its own set of coefficients.
\item Functions are not defined at element boundaries.
\item If the local functions coincide on element boundaries then $u$
  can be extended to all of $\Omega$, otherwise $u$ may be extended to
  be set-valued on the \textit{internal skeleton}
\begin{equation}
\Gamma = \bigcup_{e,f\in E^0} \left( \partial\Omega_e \cap
\partial\Omega_f \right) \setminus \partial\Omega .
\end{equation}
\end{itemize}

Typically finite element functions exhibit some form of continuity
across the element boundaries. This is achieved by (1) using specific
shape functions and (2) identifying coefficients of intersection
elements, i.e.~ when $\partial\Omega_e\cap\partial\Omega_f\neq\emptyset$.

The identification of coefficients can be written as
\begin{equation}
\hat{c}_i^e = c_{\alpha(e,i)}
\end{equation}
where $\alpha$ is the local-to-global-map.
The local-to-global-map can be constructed by identifying coefficients
with geometric entities of higher codimension, i.~e.
\begin{equation}
\hat{c}_i^e = c_{j(e,i)}^{s(e,i)}
\end{equation}
where $e'=s(e,i)\in E^{c(e,i)}$ is an entity of codimension $c(e,i)$
which is a subentity of $e$ and
$j(e,i)$ is an index within that entity, in case several
coefficients are identified with $e'$. Thus it is sufficient to
provide a map 
\begin{equation}
(e,i) \mapsto (s,c,j), \qquad e\in E^0, 0\leq i < N_e.
\end{equation}
Such a map will be called a local layout map. Using a suitable
ordering of the triples $(s,c,j)$ the local-to-global-map $\alpha$ can
be constructed. Using the identification of coefficients the class of
finite element functions treated in \texttt{dune-pdelab} can be
written as
\begin{equation}
u(x) = \sum_{e\in E^0} \left( \sum_{i=0}^{N_e-1} c_{j(e,i)}^{s(e,i)}
\hat\phi_i^e(\mu_e^{-1}(x)) \right) \chi_{\Omega_e}(x), 
\qquad x\in \bigcup_{e\in E^0} \Omega_e .
\end{equation}
Note these function may still be discontinuous at element boundaries. 
\newpage

\cite{BrennerScott}
