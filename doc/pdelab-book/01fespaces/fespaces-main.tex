\section{General construction of discrete function spaces}\label{Sec:General}

\subsection{Unconstrained Spaces}

When solving partial differential equations with grid-based
methods we deal with finite-dimensional function spaces constructed
from a given set of local functions defined on reference elements.
In the following $\Omega$ denotes an open and connected subset of
$\mathbb{R}^n$, $n\geq 1$, which is partition into a set of elements by the grid
$\mathbb{T}_h$.

\begin{definition}[Finite-dimensional function space]\label{Def:Vh}
 A finite dimensional function space $U_h(\mathbb{T}_h)$ has the
 following generic form: 
\begin{equation}\label{Eq:GenericFESpace}
\begin{split}
U_h(\mathbb{T}_h) = \Biggl\{ & u_h(x) : \bigcup_{e\in E_h^0}\Omega_e
 \to \mathbb{K}^m\,\Bigg| \\
&\quad u_h(x) = \sum_{e\in E_h^0}\sum_{i=0}^{k(e)-1} (\mathbf{u})_{g(e,i)}
\, \pi_e(\hat{x}) \, \hat\phi_{e,i}(\hat{x}) \, \chi_e(x); \, \hat{x}=\mu_e^{-1}(x) 
 \Biggr\}
\end{split}
\end{equation}
where
\begin{itemize}
\item $\mathbb{K}$ is either the set of real numbers $\mathbb{R}$ or
  the set of complex numbers $\mathbb{C}$ and $m\geq 1$ also includes
  vector-valued function spaces. 
\item $E_h^0=\{e_0, e_1, \ldots, e_{N_h^0-1}\}$ is the set of entities of
codimension 0, the elements for short. With each element we
associate a subdomain $\Omega_e$ such that $\overline{\Omega}
= \bigcup_{e\in E_h^0} \overline{\Omega}_e$ and $\Omega_e\cap\Omega_f=\emptyset$ for $e\neq f$. 
\item With each element $e$ we also associate a reference element
  $\hat\Omega_e$ and define the map $\mu_e : \overline{\hat\Omega}_e \to
  \overline{\Omega}_e$. This includes curved elements.
\item $\hat\Phi_e = \left\{\hat\phi_{e,i}\,|\,0\leq i < k(e)\right\}$ is
  the set of local basis functions for element $e$. Local basis
  functions are defined on the reference element $\hat\phi_{e,i} :
  \overline{\hat\Omega}_e \to \mathbb{R}^{m'}$.
\item $\pi_e(\hat{x})\in\mathbb{R}^{m\times m'}$ is a transformation
  needed for some finite element spaces such as the Piola
  transformation \cite{BrezziFortin} in Raviart-Thomas spaces
\begin{equation*}
\pi_e(\hat{x}) = \frac{1}{\text{det}\  \mu_e(\hat{x})} \nabla \mu_e(\hat{x})
\end{equation*}
where $\nabla \mu_e$ denotes the Jacobian of the map $\mu_e$. For most
finite element spaces $\pi_e$ is just the identity and we have $m'=m$.
\item $g : L \to \mathbb{N}$ where $L=\left\{ (e,i)\in E_h^0 \times
  \mathbb{N} \,|\, 0\leq i < k(e)\right\}$ is the local to global map
  and $\mathcal{I}_{U_h} = \text{im}\ g$ is the associated global index set.
\item $\mathbf{u}\in \mathbf{U}=\mathbb{K}^{\mathcal{I}_{U_h}}$ is a coefficient vector.
\item $\chi_e : \Omega_e \to \{0,1\}$ is the characteristic function
  of $\Omega_e$ which is one if $x\in\Omega_e$ and zero elsewhere. 
\hfill$\square$
\end{itemize}
\end{definition}

Setting $l(j) = \left\{ (e,i)\in L \,|\, g(e,i)=j \right\}$ one can
introduce global basis functions
\begin{equation*}
\phi_j(x) = \sum_{(e,i)\in l(j)} \pi_e(\hat{x}) \,
\hat\phi_{e,i}(\hat{x}) \, \chi_e(x); \ \hat{x}=\mu_e^{-1}(x).
\end{equation*} 
The global basis functions form the global basis
\begin{equation}
\Phi_{U_h} = \{\phi_i \,|\, i\in \mathcal{I}_{U_h}\}, \qquad U_h = \text{span}\ \Phi_{U_h}.
\end{equation}
Finally we define the finite element isomorphism
\begin{equation}\label{Eq:FiniteElementIsomorphism}
\text{FE}_{\Phi_{U_h}} : \mathbf{U} \to U_h, \qquad \text{FE}_{\Phi_{U_h}}(\mathbf{u}) = \sum_{i\in\mathcal{I}_{U_h}} (\mathbf{u})_i \phi_i \ .
\end{equation} 

Note that Definition \ref{Def:Vh} allows for functions that are
discontinuous at element boundaries. Formally, functions $u_h\in
U_h(\mathbb{T}_h)$ are only defined on $\bigcup_ {e\in E_h^0} \Omega_e$
but not on the skeleton $\Gamma_h = \Omega\setminus \bigcup_ {e\in
  E_h^0} \Omega_e$. It is understood that $u_h$ is extended in a
unique way to $\overline{\Omega}$ if limits are the same on $\Gamma_h$ from
the interior of the adjacent elements.

\subsection{Constrained Spaces}

Often we have the situation that problems have to be solved in a
subspace $\tilde{U}_h\subset U_h$ or even an affine subspace
$w_h+\tilde{U}_h$, where $w_h\in U_h$.

\paragraph{Basis transformation} In order to describe general
constraints we first introduce a basis transformation. So let
$\Phi_{U_h}'=\{\phi_i'\,|\, i\in \mathcal{I}_{U_h}\}$ be an alternative basis of $U_h$,
i.~e.~$U_h=\text{span}\ \Phi_{U_h}'$. The new basis functions are
given with respect to the old basis functions by the transformation
matrix $\mathbf{T}_{U_h}$:
\begin{equation}
\phi_i' = \sum_{j\in\mathcal{I}_{U_h}}
\left(\mathbf{T}_{U_h}\right)_{i,j} \phi_j, \qquad i\in \mathcal{I}_{U_h}.
\end{equation}
Formally we define a new coefficient space
$\mathbf{U}'=\mathbb{K}^{\mathcal{I}_{U_h}}$ and again have an
isomorphism with respect to the new basis:
$\text{FE}_{\Phi_{U_h}'}(\mathbf{u}') = \sum_{i\in\mathcal{I}_{U_h}}
(\mathbf{u}')_i \phi_i'$. For the representation of a function $u_h$
we get
\begin{equation}
\begin{split}
u_h = \text{FE}_{\Phi_{U_h}'}(\mathbf{u}') &= 
\sum_{j\in\mathcal{I}_{\tilde{U}_h}} (\mathbf{u}')_j \phi'_j = 
\sum_{j\in\mathcal{I}_{U_h}} (\mathbf{u}')_j \left (
\sum_{i\in\mathcal{I}_{U_h}}
\left(\mathbf{T}_{U_h}\right)_{j,i} \phi_i \right)\\
&= \sum_{i\in \mathcal{I}_{U_h}} \left (\sum_{j\in\mathcal{I}_{U_h}}
\left(\mathbf{T}^T_{U_h}\right)_{i,j} (\mathbf{u}')_j \right ) \phi_i\\
&= \text{FE}_{\Phi_{U_h}}\left( \mathbf{T}^T_{U_h} \mathbf{u}' \right) .
\end{split}
\end{equation}

\paragraph{Splitting} Subspaces are now introduced by a splitting of
the index set into unconstrained and constrained indices:
\begin{equation*}
\mathcal{I}_{U_h} = \tilde{\mathcal{I}}_{U_h} \cup
\bar{\mathcal{I}}_{U_h}, \qquad  \tilde{\mathcal{I}}_{U_h} \cap
\bar{\mathcal{I}}_{U_h} = \emptyset.
\end{equation*}
With respect to this splitting we obtain the subspaces
\begin{align*}
\tilde{U}_h' &= \text{span}\ \{\phi_i'\,|\,
i\in\tilde{\mathcal{I}}_{U_h}\}, &
\bar{U}_h' &= \text{span}\ \{\phi_i'\,|\,
i\in\bar{\mathcal{I}}_{U_h}\}
\end{align*}
and the corresponding coefficient spaces
\begin{align*}
\tilde{\mathbf{U}}' &=
\mathbb{K}^{\tilde{\mathcal{I}}_{U_h}}, &
\bar{\mathbf{U}}' &=
\mathbb{K}^{\bar{\mathcal{I}}_{U_h}},
\end{align*}
where $U_h'$ actually is the subspace where we will seek the solution
of the constrained problem.

The transformation matrix $\mathbf{T}_{U_h}$ can be written in block
form 
\begin{equation*}
\mathbf{T}_{U_h} = \left(\begin{array}{cc}
\mathbf{T}_{\tilde{U}_h,\tilde{U}_h} & \mathbf{T}_{\tilde{U}_h,\bar{U}_h}\\
\mathbf{T}_{\bar{U}_h,\tilde{U}_h} & \mathbf{T}_{\bar{U}_h,\bar{U}_h}
\end{array}\right)
\end{equation*}
according to the splitting. In the following we consider only a
special class of transformations which have the form
\begin{equation}\label{Eq:StructureTransformation}
\mathbf{T}_{U_h} = \left(\begin{array}{cc}
\mathbf{I} & \mathbf{T}_{\tilde{U}_h,\bar{U}_h}\\
\mathbf{0} & \mathbf{I}
\end{array}\right).
\end{equation}
The $\tilde{\mathcal{I}}_{U_h} \times \bar{\mathcal{I}}_{U_h}$  matrix
$\mathbf{T}_{\tilde{U}_h,\bar{U}_h}$ will usually be sparse.  If constraints only consist
of Dirichlet boundary conditions this matrix is even zero.

\paragraph{Restrictions} Below we will make use of the following restriction operators
\begin{align}
\mathbf{R}_{\tilde{\mathbf{U}}',\mathbf{U}'} &: \mathbf{U}' \to \tilde{\mathbf{U}}', & 
(\mathbf{R}_{\tilde{\mathbf{U}}',\mathbf{U}'}\mathbf{u}')_i &= 
(\mathbf{u}')_i \quad \forall i\in \tilde{\mathcal{I}}_{U_h},\\
\mathbf{R}_{\bar{\mathbf{U}}',\mathbf{U}'} &: \mathbf{U}' \to \bar{\mathbf{U}}', & 
(\mathbf{R}_{\bar{\mathbf{U}}',\mathbf{U}'}\mathbf{u}')_i &= 
(\mathbf{u}')_i \quad \forall i\in \bar{\mathcal{I}}_{U_h}.
\end{align}

 
\section{General construction of discrete problems}\label{Sec:Problems}

\subsection{Formulation in function space}

\paragraph{Unconstrained problem}

Let $r_h : U_h \times V_h \to \mathbb{K}$ be a map that is linear in its
second argument but not necessarily in its first argument. 
Then the unconstrained problem \textit{in residual formulation} reads
\begin{equation}\label{Eq:UnconstrainedProblem}
u_h\in U_h\ : \qquad r_h(u_h,v) = 0 \quad \forall v\in V_h.
\end{equation}

\paragraph{Constrained problem}

Often this problem is posed in its constrained form which reads
\begin{equation}\label{Eq:ConstrainedProblem}
u_h\in w_h + \tilde{U}'_h\ : \qquad r_h(u_h,v) = 0 \quad \forall v\in \tilde{V}'_h
\end{equation}
where the constrained spaces $\tilde{U}'_h, \tilde{V}'_h$ are
generated via a basis transformation as described above. 
We assume that Problem \eqref{Eq:ConstrainedProblem} is uniquely
solvable. 

\paragraph{Reformulation in unconstrained space}

We now seek a reformulation of Problem \eqref{Eq:ConstrainedProblem}
which seeks a solution in the unconstrained space $U_h$ and
incorporates the constraint by stating an additional constraint
equation. This is useful in practice where one does not want to change
the solution space when e.~g.~a change in boundary conditions takes
place. 

\begin{lemma}
 Let $P_h : U_h \to \tilde{U}_h$ be a projection,
i.~e.~$P_h^2 = P_h$ and assume that the affine shift is such that 
$P_h w_h = 0$. Then 
\begin{equation}\label{Eq:ConstrainedProblemReformII}
u_h\in U_h\ : \qquad \left\{\begin{array}{ll}
r_h(u_h,v) = 0 \quad \forall v\in \tilde{V}_h'\\
(I-P_h)u_h = w_h
\end{array}\right. 
\end{equation}
is equivalent to \eqref{Eq:ConstrainedProblem}.\\
\textit{Proof}. Assume that \eqref{Eq:ConstrainedProblem} holds and
$P_h w_h = 0$. Since $u_h$ solves \eqref{Eq:ConstrainedProblem}
the first equation in \eqref{Eq:ConstrainedProblemReformII} clearly holds.
Moreover, we have $u_h = w_h + \tilde{u}_h$ with $\tilde{u}_h\in
\tilde{U}_h'$ which allows us to write $u_h = w_h + P_h v_h$ for some
$v_h\in U_h$. When we can prove that $v_h=u_h$ we obtain the desired
$(I-P_h)u_h = w_h$. We now show that $v_h=u_h$:
Applying $P_h$ to both sides of the identity $u_h = w_h + P_h v_h$ yields
$P_h u_h = P_h w_h + P_h^2 v_h$. Using $P_h w_h = 0$ and $P_h^2 = P_h$
yields $P_h u_h = P_h v_h$. Thus we may identify $v_h$ and $u_h$ as
$v_h$ was arbitrary.\\
Assume now that \eqref{Eq:ConstrainedProblemReformII} holds. The first
equation of \eqref{Eq:ConstrainedProblemReformII} is the same as 
\eqref{Eq:ConstrainedProblem}. From the
second equation we conclude $u_h = w_h + P_h u_h$, i.~e.~ $u_h\in w_h
+ \tilde{U}_h'$. \hfill$\square$
\end{lemma}

\subsection{Formulation in coefficient space}

In practice the discrete problems are solved in coefficient space
using a basis representation. 

\paragraph{Unconstrained problem}

We begin with problem \eqref{Eq:UnconstrainedProblem} in the
original basis. Denote the basis of $U_h$ by
$\Phi_{U_h}=\{\phi_i\,|\, i\in\mathcal{I}_{U_h}\}$ and that of $V_h$
by $\Phi_{V_h}=\{\psi_i\,|\, i\in\mathcal{I}_{V_h}\}$.
The unconstrained problem in coefficient space results in a nonlinear
algebraic system
\begin{equation}
\begin{split}
\mathbf{u}\in\mathbf{U} : \qquad
& r_h\left(\text{FE}_{\Phi_{U_h}}(\mathbf{u}),\psi_i\right) = 0, \quad
i\in\mathcal{I}_{V_h} \\
\Leftrightarrow \  & \mathcal{R}(\mathbf{u}) = \mathbf{0}
\end{split}
\end{equation}
where we introduced the nonlinear residual map $\mathcal{R} :
\mathbf{U} \to \mathbb{K}^{\mathcal{I}_{V_h}}$ defined as 
\begin{equation}
\left(
\mathcal{R}(\mathbf{u})\right)_i =
r_h(\text{FE}_{U_h}(\mathbf{u}),\psi_i).
\end{equation}

We may introduce basis transformations $\mathbf{T}_{U_h}$,
$\mathbf{T}_{V_h}$ for trial and test space which gives us
\begin{equation}\label{Eq:TransformedUnconstrainedProblem}
\begin{split}
\mathbf{u}'\in\mathbf{U}' : \qquad 
& r_h\left(\text{FE}_{\Phi'_{U_h}}(\mathbf{u}'),\psi_i'\right) = 0, \quad
i\in\mathcal{I}_{V_h}\\
\Leftrightarrow \  &
r_h\left(\text{FE}_{\Phi_{U_h}}(\mathbf{T}^T_{U_h}\mathbf{u}'),
\sum_{j\in\mathcal{I}_{V_h}}\left(\mathbf{T}_{V_h}\right)_{i,j}\psi_j\right) = 0, \quad
i\in\mathcal{I}_{V_h}\\
\Leftrightarrow \  &
\sum_{j\in\mathcal{I}_{V_h}} \left(\mathbf{T}_{V_h}\right)_{i,j} 
r_h\left(\text{FE}_{\Phi_{U_h}}(\mathbf{T}^T_{U_h}\mathbf{u}'),
\psi_j\right) = 0, \quad
i\in\mathcal{I}_{V_h}\\
\Leftrightarrow \  &
\mathbf{T}_{V_h} \mathcal{R}\left(\mathbf{T}^T_{U_h}\mathbf{u}'\right)
= \mathbf{0} .
\end{split}
\end{equation}
Note how we used the linearity with respect to the second argument of
the residual formulation.

\paragraph{Newton solver}

We now solve \eqref{Eq:TransformedUnconstrainedProblem} using Newton's
method. Assume that a current iterate $\mathbf{u}_k'$ is given. We
seek an update $\mathbf{z}'_k$ such that $\mathbf{u}'_{k+1} = \mathbf{u}_k'
+ \mathbf{z}'_k$ and use the linearized form of the equation
\begin{equation*}
\mathbf{T}_{V_h}\mathcal{R}\left(\mathbf{T}^T_{U_h}\mathbf{u}'_{k+1}\right) \approx 
\mathbf{T}_{V_h}\mathcal{R}\left(\mathbf{T}^T_{U_h}\mathbf{u}'_{k}\right) +
\mathbf{T}_{V_h}\nabla\mathcal{R}\left(\mathbf{T}^T_{U_h}\mathbf{u}'_{k}\right) 
\mathbf{T}^T_{U_h} \mathbf{z}'_{k} = \mathbf{0}
\end{equation*}
to get the linear system for the update
\begin{equation}\label{eq:UnconstrainedUpdate}
\mathbf{T}_{V_h}\nabla\mathcal{R}\left(\mathbf{T}^T_{U_h}\mathbf{u}'_{k}\right) 
\mathbf{T}^T_{U_h} \mathbf{z}'_{k} = -
\mathbf{T}_{V_h}\mathcal{R}\left(\mathbf{T}^T_{U_h}\mathbf{u}'_{k}\right) .
\end{equation}
Here $\nabla\mathcal{R}\left(\mathbf{u}_{k}\right)$ denotes the
Jacobian matrix of the map $\mathcal{R}$. Multiplying the update
equation with $\mathbf{T}^T_{U_h}$ from the left yields
\begin{equation}\label{eq:OriginalUpdate}
\mathbf{T}^T_{U_h}\mathbf{u}'_{k+1} = \mathbf{T}^T_{U_h}\mathbf{u}_k' +
\mathbf{T}^T_{U_h}\mathbf{z}'_k .
\end{equation}
Setting $\mathbf{u}_{k} := \mathbf{T}^T_{U_h}\mathbf{u}_k'$ allows us
now to write the Newton scheme with respect to the original basis.

\begin{algo}[Newton's method for unconstrained problem]
Given the initial guess $\mathbf{u}_{0}$ iterate until convergence
\begin{enumerate}[i)]
\item Compute residual:
  $\mathbf{r}_k=-\mathcal{R}\left(\mathbf{u}_{k}\right)$.
\item Transform residual: $\mathbf{r}_k' = \mathbf{T}_{V_h}
  \mathbf{r}_k$.
\item Solve update equation:
  $\mathbf{T}_{V_h}\nabla\mathcal{R}\left(\mathbf{u}_{k}\right)  
\mathbf{T}^T_{U_h} \mathbf{z}'_{k} =  \mathbf{r}_k'$.
\item Transform update: $\mathbf{z}_{k} =
  \mathbf{T}^T_{U_h}\mathbf{z}'_k$.
\item Update: $\mathbf{u}_{k+1} = \mathbf{u}_k
+ \mathbf{z}_k$.
\end{enumerate}
Note that two applications of the basis transformation, for the
residual and the update, are necessary in each step. These
transformations are cheap due to the structure of the transformation
\eqref{Eq:StructureTransformation}. In step (iii) the transformed
Jacobian system is required. \hfill$\square$
\end{algo}

\paragraph{Constrained problem} We now seek to solve problem
\eqref{Eq:ConstrainedProblemReformII} in coefficient space. The
projection $P_h$ is taken from the follwing commutative diagram:
\begin{equation*}
\begin{CD}
U_h @>{P_h = \text{FE}_{\Phi'_{U_h}}
\mathbf{R}^T_{\tilde{\mathbf{U}}',\mathbf{U}'}
\mathbf{R}_{\tilde{\mathbf{U}}',\mathbf{U}'} 
\text{FE}_{\Phi'_{U_h}}^{-1}}>> \tilde{U}_h'\\
@A{\text{FE}_{\Phi'_{U_h}}}AA @AA{\text{FE}_{\Phi'_{U_h}}
\mathbf{R}^T_{\tilde{\mathbf{U}}'\mathbf{U}'}}A\\
\mathbf{U}' @>{\qquad\mathbf{R}_{\tilde{\mathbf{U}}',\mathbf{U}'}\qquad}>> \tilde{\mathbf{U}}' 
\end{CD}
\end{equation*}
Using this definition of the projection we obtain for the second
equation in \eqref{Eq:ConstrainedProblemReformII}  
\begin{equation}\label{Eq:SideConditionCoefficient}
\begin{split}
&(I-P_h) u_h = w_h \\
\Leftrightarrow \quad & 
\left(\text{FE}_{\Phi'_{U_h}}\text{FE}_{\Phi'_{U_h}}^{-1}
- \text{FE}_{\Phi'_{U_h}}
\mathbf{R}^T_{\tilde{\mathbf{U}}',\mathbf{U}'}
\mathbf{R}_{\tilde{\mathbf{U}}',\mathbf{U}'} 
\text{FE}_{\Phi'_{U_h}}^{-1}\right)\text{FE}_{\Phi'_{U_h}}\mathbf{u}'
= \text{FE}_{\Phi'_{U_h}}
\mathbf{R}^T_{\bar{\mathbf{U}}',\mathbf{U}'} \mathbf{w}'\\
\Leftrightarrow \quad &
\left( \mathbf{I} - \mathbf{R}^T_{\tilde{\mathbf{U}}',\mathbf{U}'}
\mathbf{R}_{\tilde{\mathbf{U}}',\mathbf{U}'}\right) \mathbf{u}' =
\mathbf{R}^T_{\bar{\mathbf{U}}',\mathbf{U}'} \mathbf{w}' \\
\Leftrightarrow \quad &
\mathbf{R}^T_{\bar{\mathbf{U}}',\mathbf{U}'}
\mathbf{R}_{\bar{\mathbf{U}}',\mathbf{U}'} \mathbf{u}' =
\mathbf{R}^T_{\bar{\mathbf{U}}',\mathbf{U}'} \mathbf{w}'\\
\Leftrightarrow \quad &
\mathbf{R}_{\bar{\mathbf{U}}',\mathbf{U}'} \mathbf{u}' = \mathbf{w}' .
\end{split}
\end{equation}
For the first equation in \eqref{Eq:ConstrainedProblemReformII} 
we proceed as in \eqref{Eq:TransformedUnconstrainedProblem}
\begin{equation}\label{Eq:TransformedConstrainedProblem2}
\begin{split}
\mathbf{u}'\in\mathbf{U}' : \qquad 
& r_h\left(\text{FE}_{\Phi'_{U_h}}(\mathbf{u}'),\psi_i'\right) = 0, \quad
i\in\tilde{\mathcal{I}}_{V_h}\\
\Leftrightarrow \  &
r_h\left(\text{FE}_{\Phi_{U_h}}(\mathbf{T}^T_{U_h}\mathbf{u}'),
\sum_{j\in\mathcal{I}_{V_h}}\left(\mathbf{T}_{V_h}\right)_{i,j}\psi_j\right) = 0, \quad
i\in\tilde{\mathcal{I}}_{V_h}\\
\Leftrightarrow \  &
\sum_{j\in\mathcal{I}_{V_h}} \left(\mathbf{T}_{V_h}\right)_{i,j} 
r_h\left(\text{FE}_{\Phi_{U_h}}(\mathbf{T}^T_{U_h}\mathbf{u}'),
\psi_j\right) = 0, \quad
i\in\tilde{\mathcal{I}}_{V_h}\\
\Leftrightarrow \  &
\underbrace{\left(\mathbf{R}_{\tilde{\mathbf{V}}',\mathbf{V}'} +
\mathbf{T}_{\tilde{V}_h,\bar{V}_h}\mathbf{R}_{\bar{\mathbf{V}}',\mathbf{V}'}
\right)}_{\mathbf{S}_{\tilde{\mathbf{V}}'}}\mathcal{R}\left(\mathbf{T}^T_{U_h}\mathbf{u}'\right)=
\mathbf{S}_{\tilde{\mathbf{V}}'} \mathcal{R}\left(\mathbf{T}^T_{U_h}\mathbf{u}'\right)
= \mathbf{0} .
\end{split}
\end{equation}
Here we made use of \eqref{Eq:StructureTransformation} in the final line.

The equivalent formulation of
\eqref{Eq:ConstrainedProblemReformII} in coefficient space then reads
\begin{equation}\label{Eq:ConstrainedProblemInCoefficientSpace}
\mathbf{u}'\in\mathbf{U}' : \qquad \left\{\begin{array}{rcl}
\mathbf{S}_{\tilde{\mathbf{V}}'}
\mathcal{R}\left(\mathbf{T}^T_{U_h}\mathbf{u}'\right)
& = & \mathbf{0}\\
\mathbf{R}_{\bar{\mathbf{U}}',\mathbf{U}'} \mathbf{u}' & = & \mathbf{w}'
\end{array}\right. 
\end{equation}
The idea in this formulation is that with respect to the transformed
basis the affine shift (for Dirichlet boundary conditions) can be
``encoded'' in the constrained degrees of freedom
$\mathbf{R}_{\bar{\mathbf{U}}',\mathbf{U}'} \mathbf{u}'$. This is
possible because the subspace $\tilde{U}_h'$ is the image
of the unconstrained degrees of freedom
$\mathbf{R}_{\tilde{\mathbf{U}}',\mathbf{U}'} \mathbf{u}'$ 
and the decomposition is orthogonal
(i.~e.~$\mathbf{R}^T_{\bar{\mathbf{U}}',\mathbf{U}'}
\mathbf{R}_{\bar{\mathbf{U}}',\mathbf{U}'}$ and
$\mathbf{R}^T_{\tilde{\mathbf{U}}',\mathbf{U}'}
\mathbf{R}_{\tilde{\mathbf{U}}',\mathbf{U}'}$ are orthogonal
projections). 





\paragraph{Newton's method for constrained problem}

Now we need to solve \eqref{Eq:ConstrainedProblemInCoefficientSpace}
with Newton's method. Let some $\mathbf{u}_{k}'$, set
$\mathbf{u}_{k+1}' = \mathbf{u}_{k}' + \mathbf{z}_{k}'$ and
seek an equation for the update $\mathbf{z}_{k}'$. 

Inserting $\mathbf{u}_{k+1}'$ into the second equation of
\eqref{Eq:ConstrainedProblemInCoefficientSpace} yields
\begin{equation}\label{Eq:SideCond}
\begin{split}
& \mathbf{R}_{\bar{\mathbf{U}}',\mathbf{U}'} \mathbf{u}_{k+1}'
= \mathbf{R}_{\bar{\mathbf{U}}',\mathbf{U}'} \mathbf{u}_{k}' +
\mathbf{R}_{\bar{\mathbf{U}}',\mathbf{U}'} \mathbf{z}_{k}'
 =  \bar{\mathbf{w}}'\\ 
\Leftrightarrow\qquad &
\mathbf{R}_{\bar{\mathbf{U}}',\mathbf{U}'} \mathbf{z}_{k}' = 
\bar{\mathbf{w}}' - \mathbf{R}_{\bar{\mathbf{U}}',\mathbf{U}'}
\mathbf{u}_{k}' = \mathbf{0}\\
\Leftrightarrow\qquad &
\bar{\mathbf{z}}_{k}' = \mathbf{0}
\end{split}
\end{equation}
where we introduced $\mathbf{z}_{k}' =
\mathbf{R}^T_{\bar{\mathbf{U}}',\mathbf{U}'}
\bar{\mathbf{z}}_{k}'$ and used $\mathbf{R}_{\bar{\mathbf{U}}',\mathbf{U}'}
\mathbf{R}^T_{\bar{\mathbf{U}}',\mathbf{U}'}=\mathbf{I}$. 
Note that the affine shift is not changed during the iteration
\begin{equation}
\mathbf{R}_{\bar{\mathbf{U}}',\mathbf{U}'}\mathbf{u}_{k+1}' =
\mathbf{R}_{\bar{\mathbf{U}}',\mathbf{U}'} \mathbf{u}_{k}' +
\underbrace{\mathbf{R}_{\bar{\mathbf{U}}',\mathbf{U}'}
  \mathbf{z}_{k}'}_{= \mathbf{0}} =
\mathbf{R}_{\bar{\mathbf{U}}',\mathbf{U}'} \mathbf{u}_{k}'
\end{equation}
and it is thus sufficient to satisfy the affine shift in the initial
guess $\mathbf{R}_{\bar{\mathbf{U}}',\mathbf{U}'} \mathbf{u}_{0}' =
\bar{\mathbf{w}}'$.

Now insert  $\mathbf{u}_{k+1}'$ into the first equation of
\eqref{Eq:ConstrainedProblemInCoefficientSpace}:
\begin{equation}
\begin{split}
\mathbf{S}_{\tilde{\mathbf{V}}'}
&\mathcal{R}\left(\mathbf{T}^T_{U_h}\mathbf{u}_{k+1}'\right) \\
&= \mathbf{S}_{\tilde{\mathbf{V}}'}
\mathcal{R}\left(\mathbf{T}^T_{U_h}\mathbf{u}_{k}' +
\mathbf{T}^T_{U_h}\mathbf{z}_{k}' \right)\\
&= \mathbf{S}_{\tilde{\mathbf{V}}'}
\mathcal{R}\left(\mathbf{T}^T_{U_h}\mathbf{u}_{k}' +
\mathbf{T}^T_{U_h} \left(\mathbf{R}^T_{\bar{\mathbf{U}}',\mathbf{U}'}
\underbrace{\mathbf{R}_{\bar{\mathbf{U}}',\mathbf{U}'}
  \mathbf{z}_{k}'}_{=\mathbf{0}, \text{ cf.\eqref{Eq:SideCond}}} +
\mathbf{R}^T_{\tilde{\mathbf{U}}',\mathbf{U}'} 
\mathbf{R}_{\tilde{\mathbf{U}}',\mathbf{U}'} \mathbf{z}_{k}' \right)
\right)\\
&= \mathbf{S}_{\tilde{\mathbf{V}}'}
\mathcal{R}\left(\mathbf{T}^T_{U_h}\mathbf{u}_{k}' +
\mathbf{T}^T_{U_h} \mathbf{R}^T_{\tilde{\mathbf{U}}',\mathbf{U}'} 
\mathbf{R}_{\tilde{\mathbf{U}}',\mathbf{U}'} \mathbf{z}_{k}' \right)\\
&= \mathbf{S}_{\tilde{\mathbf{V}}'}
\mathcal{R}\left(\mathbf{T}^T_{U_h}\mathbf{u}_{k}' +
\underbrace{\left(\mathbf{R}^T_{\tilde{\mathbf{U}}',\mathbf{U}'} +
\mathbf{R}^T_{\bar{\mathbf{U}}',\mathbf{U}'} \mathbf{T}^T_{\tilde{U}_h,\bar{U}_h}
\right)}_{=: \,\mathbf{S}^T_{\tilde{\mathbf{U}}'}}
\mathbf{R}_{\tilde{\mathbf{U}}',\mathbf{U}'} \mathbf{z}_{k}' 
\right) \\
&= 
\mathbf{S}_{\tilde{\mathbf{V}}'}
\mathcal{R}\left(\mathbf{T}^T_{U_h}\mathbf{u}_{k}' + 
\mathbf{S}^T_{\tilde{\mathbf{U}}'}
\mathbf{R}_{\tilde{\mathbf{U}}',\mathbf{U}'}
\mathbf{R}^T_{\tilde{\mathbf{U}}',\mathbf{U}'} \tilde{\mathbf{z}}_{k}'
\right)
=
\mathbf{S}_{\tilde{\mathbf{V}}'}
\mathcal{R}\left(\mathbf{T}^T_{U_h}\mathbf{u}_{k}' + 
\mathbf{S}^T_{\tilde{\mathbf{U}}'} \tilde{\mathbf{z}}_{k}'\right)
\end{split}
\end{equation}
where we introduced $\mathbf{z}_{k}' =
\mathbf{R}^T_{\tilde{\mathbf{U}}',\mathbf{U}'}
\tilde{\mathbf{z}}_{k}'$ and used $\mathbf{R}_{\tilde{\mathbf{U}}',\mathbf{U}'}
\mathbf{R}^T_{\tilde{\mathbf{U}}',\mathbf{U}'}=\mathbf{I}$. 
Linearization now gives
\begin{equation}
\mathbf{S}_{\tilde{\mathbf{V}}'}
\mathcal{R}\left(\mathbf{T}^T_{U_h}\mathbf{u}_{k}' + 
\mathbf{S}^T_{\tilde{\mathbf{U}}'} \tilde{\mathbf{z}}_{k}'\right)
\approx \mathbf{S}_{\tilde{\mathbf{V}}'}
\mathcal{R}\left(\mathbf{T}^T_{U_h}\mathbf{u}_{k}'\right) 
+ \mathbf{S}_{\tilde{\mathbf{V}}'} \nabla
\mathcal{R}\left(\mathbf{T}^T_{U_h}\mathbf{u}_{k}'\right)
\mathbf{S}^T_{\tilde{\mathbf{U}}'} \tilde{\mathbf{z}}_{k}' = \mathbf{0}.
\end{equation}
Thus the equation for the update reads
\begin{equation}
\mathbf{S}_{\tilde{\mathbf{V}}'} \nabla
\mathcal{R}\left(\mathbf{T}^T_{U_h}\mathbf{u}_{k}'\right)
\mathbf{S}^T_{\tilde{\mathbf{U}}'} \tilde{\mathbf{z}}_{k}'
= - \mathbf{S}_{\tilde{\mathbf{V}}'}
\mathcal{R}\left(\mathbf{T}^T_{U_h}\mathbf{u}_{k}'\right) .
\end{equation}

We are now in a position to give the full algorithm for the
constrained problem. As in the unconstrained version we make use of
\eqref{eq:OriginalUpdate} to hold the iterates with respect to the
original basis and only compute the update in the transformed basis. 

\begin{algo}[Newton's method for constrained problem]
Let the initial guess $\mathbf{u}_{0}$ with
$\text{FE}_{\Phi_{U_h}}(\mathbf{u}_{0}) \in w_h + \tilde{U}_h'$ be given. 
Iterate until convergence
\begin{enumerate}[i)]
\item Compute residual:
  $\mathbf{r}_k=-\mathcal{R}\left(\mathbf{u}_{k}\right)$.
\item Transform residual: $\mathbf{r}_k' = \mathbf{S}_{\tilde{\mathbf{V}}'}
  \mathbf{r}_k$.
\item Solve update equation:
\begin{equation*}
\left(\begin{array}{cc}
\mathbf{S}_{\tilde{\mathbf{V}}'} \nabla
\mathcal{R}\left(\mathbf{T}^T_{U_h}\mathbf{u}_{k}'\right)
\mathbf{S}^T_{\tilde{\mathbf{U}}'} & \mathbf{0}\\
\mathbf{0} & \mathbf{I}
\end{array}\right) 
\left(\begin{array}{c}
\tilde{\mathbf{z}}_{k}'\\
\bar{\mathbf{z}}_{k}'
\end{array}\right) =
\left(\begin{array}{c}
\mathbf{r}_k'\\
\mathbf{0}
\end{array}\right) 
\end{equation*}
and set $\mathbf{z}'_{k} =
\mathbf{R}^T_{\tilde{\mathbf{U}}',\mathbf{U}'} \tilde{\mathbf{z}}_{k}'$.
\item Transform update: $\mathbf{z}_{k} =
  \mathbf{T}^T_{U_h}\mathbf{z}'_k$. (This is where interpolation to
  hanging nodes is done).
\item Update: $\mathbf{u}_{k+1} = \mathbf{u}_k
+ \mathbf{z}_k$. \hfill$\square$
\end{enumerate}
\end{algo}


\section{Local functions}

\section{Grid function space}

\section{Constraints}

\section{Composite function spaces}
