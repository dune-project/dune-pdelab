\documentclass{scrartcl}

\usepackage{dsfont}
\usepackage{amsmath}
\usepackage{tikz}

\usetikzlibrary{calc}

\newenvironment{Matrix}[1]{\left(\mspace{-6mu}\begin{array}{*{#1}{r}}}{\end{array}\mspace{-6mu}\right)}
\newenvironment{Matrix3}{\begin{Matrix}3}{\end{Matrix}}
\newenvironment{Matrix6}{\begin{Matrix}6}{\end{Matrix}}
\newenvironment{Vector}{\begin{Matrix}1}{\end{Matrix}}

\newcommand\psqrt{\phantom{\sqrt0}}

\begin{document}

\title{The Electrodynamic Test Problem}

\maketitle

\section{Domain: Kuhn Triangulated Cube}

The domain is a $1\times1\times1$ cube with the vertices
\begin{align}
  \xi^0&=\begin{pmatrix}0\\0\\0\end{pmatrix} &
  \xi^1&=\begin{pmatrix}1\\0\\0\end{pmatrix} &
  \xi^2&=\begin{pmatrix}0\\1\\0\end{pmatrix} &
  \xi^3&=\begin{pmatrix}1\\1\\0\end{pmatrix} \\
  \xi^4&=\begin{pmatrix}0\\0\\1\end{pmatrix} &
  \xi^5&=\begin{pmatrix}1\\0\\1\end{pmatrix} &
  \xi^6&=\begin{pmatrix}0\\1\\1\end{pmatrix} &
  \xi^7&=\begin{pmatrix}1\\1\\1\end{pmatrix}.
\end{align}
It is triangulated into 6 tetrahedrons with the vertices
\begin{align}
  \mathbf x^0_{\Omega_0}&=\xi^0 &
  \mathbf x^1_{\Omega_0}&=\xi^1 &
  \mathbf x^2_{\Omega_0}&=\xi^3 &
  \mathbf x^3_{\Omega_0}&=\xi^7 \\
  \mathbf x^0_{\Omega_1}&=\xi^0 &
  \mathbf x^1_{\Omega_1}&=\xi^2 &
  \mathbf x^2_{\Omega_1}&=\xi^3 &
  \mathbf x^3_{\Omega_1}&=\xi^7 \\
  \mathbf x^0_{\Omega_2}&=\xi^0 &
  \mathbf x^1_{\Omega_2}&=\xi^2 &
  \mathbf x^2_{\Omega_2}&=\xi^6 &
  \mathbf x^3_{\Omega_2}&=\xi^7 \\
  \mathbf x^0_{\Omega_3}&=\xi^0 &
  \mathbf x^1_{\Omega_3}&=\xi^4 &
  \mathbf x^2_{\Omega_3}&=\xi^6 &
  \mathbf x^3_{\Omega_3}&=\xi^7 \\
  \mathbf x^0_{\Omega_4}&=\xi^0 &
  \mathbf x^1_{\Omega_4}&=\xi^4 &
  \mathbf x^2_{\Omega_4}&=\xi^5 &
  \mathbf x^3_{\Omega_4}&=\xi^7 \\
  \mathbf x^0_{\Omega_5}&=\xi^0 &
  \mathbf x^1_{\Omega_5}&=\xi^1 &
  \mathbf x^2_{\Omega_5}&=\xi^5 &
  \mathbf x^3_{\Omega_5}&=\xi^7.
\end{align}
There are the following 19 edges
\begin{align}
  \mathcal E^0   &=\overline{\xi^0\xi^1} &
  \mathcal E^1   &=\overline{\xi^0\xi^2} &
  \mathcal E^2   &=\overline{\xi^0\xi^3} &
  \mathcal E^3   &=\overline{\xi^0\xi^4} \\
  \mathcal E^4   &=\overline{\xi^0\xi^5} &
  \mathcal E^5   &=\overline{\xi^0\xi^6} &
  \mathcal E^6   &=\overline{\xi^0\xi^7} &
  \mathcal E^7   &=\overline{\xi^1\xi^3} \\
  \mathcal E^8   &=\overline{\xi^1\xi^5} &
  \mathcal E^9   &=\overline{\xi^1\xi^7} &
  \mathcal E^{10}&=\overline{\xi^2\xi^3} &
  \mathcal E^{11}&=\overline{\xi^2\xi^6} \\
  \mathcal E^{12}&=\overline{\xi^2\xi^7} &
  \mathcal E^{13}&=\overline{\xi^3\xi^7} &
  \mathcal E^{14}&=\overline{\xi^4\xi^5} &
  \mathcal E^{15}&=\overline{\xi^4\xi^6} \\
  \mathcal E^{16}&=\overline{\xi^4\xi^7} &
  \mathcal E^{17}&=\overline{\xi^5\xi^7} &
  \mathcal E^{18}&=\overline{\xi^6\xi^7}.
\end{align}
The mapping of local and global edge numbers is as follows:
\begin{center}\begin{tabular}{c|c|c|c|c|c|c}
Tet & Edge 0 & Edge 1 & Edge 2 & Edge 3 & Edge 4 & Edge 5 \\
\hline
 0  &    0   &    2   &    7   &    6   &    9   &   13   \\
 1  &    1   &    2   &   10   &    6   &   12   &   13   \\
 2  &    1   &    5   &   11   &    6   &   12   &   18   \\
 3  &    3   &    5   &   15   &    6   &   16   &   18   \\
 4  &    3   &    4   &   14   &    6   &   16   &   17   \\
 5  &    0   &    4   &    8   &    6   &    9   &   17
\end{tabular}\end{center}

\section{Boundary Conditions}

We use PEC boundary conditions everywhere, i.e. $\mathbf{\hat n}\times\mathbf
E=0$ on $\partial\Omega$.  That leaves only one non-constrained degree of
freedom: 6, which coincides with edge 3 in all tets.

\section{Matrix}

The residual form is
\begin{equation}
  r=T(u^{n+1}-2u^n+u^{n-1})+(\Delta t)^2Su^{n-1}\stackrel!=0,
\end{equation}
which leads to the linear equation system
\begin{align}
  Ax&=b \\
  A_{ij}&=\begin{cases}
    T_{ij}     &\text{for both $i$ and $j$ non-constrained} \\
    \delta_{ij}&\text{otherwise}
  \end{cases} \\
  b_i&=
  \begin{cases}
    (2Tu^n-(T+(\Delta t)^2S)u^{n-1})_i&\text{for $i$ non-constrained} \\
    0                                &\text{otherwise}
  \end{cases}
\end{align}
For $A$ we are currently only interested in the entry $A_{66}$ which is the
non-constrained one.
\begin{equation}
  T_{ij}=\int_\Omega\epsilon\psi_i\cdot\psi_jdV
\end{equation}

\section{Shape Functions}

So we need $\psi_6=\sum_{i=0}^5\hat\psi^i_3\chi^i$.

\subsection{\ldots in $\Omega_0$}
\begin{align}
  \mathbf x^0_{\Omega_0}&=\begin{pmatrix}0\\0\\0\end{pmatrix} &
  \mathbf x^1_{\Omega_0}&=\begin{pmatrix}1\\0\\0\end{pmatrix} &
  \mathbf x^2_{\Omega_0}&=\begin{pmatrix}1\\1\\0\end{pmatrix} &
  \mathbf x^3_{\Omega_0}&=\begin{pmatrix}1\\1\\1\end{pmatrix}
\end{align}
This yields
\begin{equation}
  M_{\Omega_0}=\begin{Matrix6}
     0 &  0 &  0 & 1 & 0 & 0 \\
     0 &  0 &  0 & 1 & 1 & 0 \\
    -1 &  0 &  0 & 0 & 1 & 0 \\
     0 &  0 &  0 & 1 & 1 & 1 \\
    -1 & -1 &  0 & 0 & 1 & 1 \\
     0 & -1 & -1 & 0 & 0 & 1
  \end{Matrix6},
\end{equation}
and, with the help of maxima
\begin{equation}
  M_{\Omega_0}^{-1}=\begin{Matrix6}
    -1 &  1 & -1 & 0 &  0 &  0 \\
     0 & -1 &  1 & 1 & -1 &  0 \\
     0 &  0 & -1 & 0 &  1 & -1 \\
     1 &  0 &  0 & 0 &  0 &  0 \\
    -1 &  1 &  0 & 0 &  0 &  0 \\
     0 & -1 &  0 & 1 &  0 &  0
  \end{Matrix6}.
\end{equation}
I'm primarily interested in the shape function corresponding to the long
diagonal, which has the right hand side
\begin{align}
  r_{\Omega_0}^0&=\begin{pmatrix} 1 \\ 0 \\ 0 \\ 0 \\ 0 \\ 0 \end{pmatrix} &
  r_{\Omega_0}^1&=\begin{pmatrix} 0 \\ \sqrt2 \\ 0 \\ 0 \\ 0 \\ 0 \end{pmatrix} &
  r_{\Omega_0}^2&=\begin{pmatrix} 0 \\ 0 \\ 1 \\ 0 \\ 0 \\ 0 \end{pmatrix} \\
  r_{\Omega_0}^3&=\begin{pmatrix} 0 \\ 0 \\ 0 \\ \sqrt3 \\ 0 \\ 0 \end{pmatrix} &
  r_{\Omega_0}^4&=\begin{pmatrix} 0 \\ 0 \\ 0 \\ 0 \\ \sqrt2 \\ 0 \end{pmatrix} &
  r_{\Omega_0}^5&=\begin{pmatrix} 0 \\ 0 \\ 0 \\ 0 \\ 0 \\ 1 \end{pmatrix}.
\end{align}
This yields the coefficients
\begin{align}
  \mathrm A_{\Omega_0}^0&=\psqrt\begin{Matrix3}
     0 & -1 &  0 \\
     1 &  0 &  0 \\
     0 &  0 &  0
  \end{Matrix3} & 
  \mathbf a_{\Omega_0}^0&=\psqrt\begin{Vector}  1 \\ -1 \\  0 \end{Vector}
\\
  \mathrm A_{\Omega_0}^1&=\sqrt2\begin{Matrix3}
     0 &  1 & -1 \\
    -1 &  0 &  0 \\
     1 &  0 &  0
  \end{Matrix3} & 
  \mathbf a_{\Omega_0}^1&=\sqrt2\begin{Vector}  0 \\  1 \\ -1 \end{Vector}
\\
  \mathrm A_{\Omega_0}^2&=\psqrt\begin{Matrix3}
     0 & -1 &  1 \\
     1 &  0 & -1 \\
    -1 &  1 &  0
  \end{Matrix3} & 
  \mathbf a_{\Omega_0}^2&=\psqrt\begin{Vector}  0 \\  0 \\  0 \end{Vector}
\\
  \mathrm A_{\Omega_0}^3&=\sqrt3\begin{Matrix3}
     0 &  0 &  1 \\
     0 &  0 &  0 \\
    -1 &  0 &  0
  \end{Matrix3} & 
  \mathbf a_{\Omega_0}^3&=\sqrt3\begin{Vector}  0 \\  0 \\  1 \end{Vector}
\\
  \mathrm A_{\Omega_0}^4&=\sqrt2\begin{Matrix3}
     0 &  0 & -1 \\
     0 &  0 &  1 \\
     1 & -1 &  0
  \end{Matrix3} & 
  \mathbf a_{\Omega_0}^4&=\sqrt2\begin{Vector}  0 \\  0 \\  0 \end{Vector}
\\
  \mathrm A_{\Omega_0}^5&=\psqrt\begin{Matrix3}
     0 &  0 &  0 \\
     0 &  0 & -1 \\
     0 &  1 &  0
  \end{Matrix3} & 
  \mathbf a_{\Omega_0}^5&=\psqrt\begin{Vector}  0 \\  0 \\  0 \end{Vector}
\end{align}
with the resulting shape functions
\begin{align}
  \psi_{\Omega_0}^0(\mathbf x)
    &=\psqrt\begin{pmatrix} 1-x_1 \\ x_0-1 \\ 0 \end{pmatrix} &
  \nabla\times\psi_{\Omega_0}^0
    &=\psqrt\begin{Vector}  0 \\  0 \\  2 \end{Vector}
\\
  \psi_{\Omega_0}^1(\mathbf x)
    &=\sqrt2\begin{pmatrix} x_1-x_2 \\ 1-x_0 \\ x_0-1 \end{pmatrix} &
  \nabla\times\psi_{\Omega_0}^1
    &=\sqrt2\begin{Vector}  0 \\ -2 \\ -2 \end{Vector}
\\
  \psi_{\Omega_0}^2(\mathbf x)
    &=\psqrt\begin{pmatrix} x_2-x_1 \\ x_0-x_2 \\ x_1-x_0 \end{pmatrix} &
  \nabla\times\psi_{\Omega_0}^2
    &=\psqrt\begin{Vector}  2 \\  2 \\  2 \end{Vector}
\\
  \psi_{\Omega_0}^3(\mathbf x)
    &=\sqrt3\begin{pmatrix} x_2 \\ 0 \\ 1-x_0 \end{pmatrix} &
  \nabla\times\psi_{\Omega_0}^3
    &=\sqrt3\begin{Vector}  0 \\  2 \\  0 \end{Vector}
\\
  \psi_{\Omega_0}^4(\mathbf x)
    &=\sqrt2\begin{pmatrix} -x_2 \\ x_2 \\ x_0-x_1 \end{pmatrix} &
  \nabla\times\psi_{\Omega_0}^4
    &=\sqrt2\begin{Vector} -2 \\ -2 \\  0 \end{Vector}
\\
  \psi_{\Omega_0}^5(\mathbf x)
    &=\psqrt\begin{pmatrix} 0 \\ -x_2 \\ x_1 \end{pmatrix} &
  \nabla\times\psi_{\Omega_0}^5
    &=\psqrt\begin{Vector}  2 \\  0 \\  0 \end{Vector}
\end{align}
To integrate arbitrary second order functions over tetrahedron 0 we use
$\int_{\Omega_0}dV\cdots=\int_0^1dx_0\int_0^{x_0}dx_1\int_0^{x_1}dx_2\cdots$.
That yields (with the help of maxima) the following contributions
\begin{align}
  T_{ 0, 0;\Omega_0}=\int_{\Omega_0}dV(\psi^0_{\Omega_0})^2
  &=\frac1{15} &
  S_{ 0, 0;\Omega_0}=\int_{\Omega_0}dV(\nabla\times\psi^0_{\Omega_0})^2
  &=\frac23 \\
  T_{ 2, 2;\Omega_0}=\int_{\Omega_0}dV(\psi^1_{\Omega_0})^2
  &=\frac1{10} &
  S_{ 2, 2;\Omega_0}=\int_{\Omega_0}dV(\nabla\times\psi^1_{\Omega_0})^2
  &=\frac83 \\
  T_{ 7, 7;\Omega_0}=\int_{\Omega_0}dV(\psi^2_{\Omega_0})^2
  &=\frac1{12} &
  S_{ 7, 7;\Omega_0}=\int_{\Omega_0}dV(\nabla\times\psi^2_{\Omega_0})^2
  &=2 \\
  T_{ 6, 6;\Omega_0}=\int_{\Omega_0}dV(\psi^3_{\Omega_0})^2
  &=\frac1{10} &
  S_{ 6, 6;\Omega_0}=\int_{\Omega_0}dV(\nabla\times\psi^3_{\Omega_0})^2
  &=2 \\
  T_{ 9, 9;\Omega_0}=\int_{\Omega_0}dV(\psi^4_{\Omega_0})^2
  &=\frac1{10} &
  S_{ 9, 9;\Omega_0}=\int_{\Omega_0}dV(\nabla\times\psi^4_{\Omega_0})^2
  &=\frac83 \\
  T_{13,13;\Omega_0}=\int_{\Omega_0}dV(\psi^5_{\Omega_0})^2
  &=\frac1{15} &
  S_{13,13;\Omega_0}=\int_{\Omega_0}dV(\nabla\times\psi^5_{\Omega_0})^2
  &=\frac23
\end{align}

Assuming the other tetrahedra yield the same figure (which is likely because
of symmetry) we get $T_{66}=0.6$, which is just what the code yields as well.
So this seems to be correct.

\subsection{\ldots in $\Omega_1$}
\begin{align}
  \mathbf x^0_{\Omega_0}&=\begin{pmatrix}0\\0\\0\end{pmatrix} &
  \mathbf x^1_{\Omega_0}&=\begin{pmatrix}0\\1\\0\end{pmatrix} &
  \mathbf x^2_{\Omega_0}&=\begin{pmatrix}1\\1\\0\end{pmatrix} &
  \mathbf x^3_{\Omega_0}&=\begin{pmatrix}1\\1\\1\end{pmatrix}
\end{align}
This yields
\begin{equation}
  M_{\Omega_0}=\begin{Matrix6}
     0 &  0 &  0 & 1 & 0 & 0 \\
     0 &  0 &  0 & 1 & 1 & 0 \\
    -1 &  0 &  0 & 0 & 1 & 0 \\
     0 &  0 &  0 & 1 & 1 & 1 \\
    -1 & -1 &  0 & 0 & 1 & 1 \\
     0 & -1 & -1 & 0 & 0 & 1
  \end{Matrix6},
\end{equation}
and, with the help of maxima
\begin{equation}
  M_{\Omega_0}^{-1}=\begin{Matrix6}
    -1 &  1 & -1 & 0 &  0 &  0 \\
     0 & -1 &  1 & 1 & -1 &  0 \\
     0 &  0 & -1 & 0 &  1 & -1 \\
     1 &  0 &  0 & 0 &  0 &  0 \\
    -1 &  1 &  0 & 0 &  0 &  0 \\
     0 & -1 &  0 & 1 &  0 &  0
  \end{Matrix6}.
\end{equation}
I'm primarily interested in the shape function corresponding to the long
diagonal, which has the right hand side
\begin{align}
  r_{\Omega_0}^0&=\begin{pmatrix} 1 \\ 0 \\ 0 \\ 0 \\ 0 \\ 0 \end{pmatrix} &
  r_{\Omega_0}^1&=\begin{pmatrix} 0 \\ \sqrt2 \\ 0 \\ 0 \\ 0 \\ 0 \end{pmatrix} &
  r_{\Omega_0}^2&=\begin{pmatrix} 0 \\ 0 \\ 1 \\ 0 \\ 0 \\ 0 \end{pmatrix} \\
  r_{\Omega_0}^3&=\begin{pmatrix} 0 \\ 0 \\ 0 \\ \sqrt3 \\ 0 \\ 0 \end{pmatrix} &
  r_{\Omega_0}^4&=\begin{pmatrix} 0 \\ 0 \\ 0 \\ 0 \\ \sqrt2 \\ 0 \end{pmatrix} &
  r_{\Omega_0}^5&=\begin{pmatrix} 0 \\ 0 \\ 0 \\ 0 \\ 0 \\ 1 \end{pmatrix}.
\end{align}
This yields the coefficients
\begin{align}
  \mathrm A_{\Omega_0}^0&=\psqrt\begin{Matrix3}
     0 & -1 &  0 \\
     1 &  0 &  0 \\
     0 &  0 &  0
  \end{Matrix3} & 
  \mathbf a_{\Omega_0}^0&=\psqrt\begin{Vector}  1 \\ -1 \\  0 \end{Vector}
\\
  \mathrm A_{\Omega_0}^1&=\sqrt2\begin{Matrix3}
     0 &  1 & -1 \\
    -1 &  0 &  0 \\
     1 &  0 &  0
  \end{Matrix3} & 
  \mathbf a_{\Omega_0}^1&=\sqrt2\begin{Vector}  0 \\  1 \\ -1 \end{Vector}
\\
  \mathrm A_{\Omega_0}^2&=\psqrt\begin{Matrix3}
     0 & -1 &  1 \\
     1 &  0 & -1 \\
    -1 &  1 &  0
  \end{Matrix3} & 
  \mathbf a_{\Omega_0}^2&=\psqrt\begin{Vector}  0 \\  0 \\  0 \end{Vector}
\\
  \mathrm A_{\Omega_0}^3&=\sqrt3\begin{Matrix3}
     0 &  0 &  1 \\
     0 &  0 &  0 \\
    -1 &  0 &  0
  \end{Matrix3} & 
  \mathbf a_{\Omega_0}^3&=\sqrt3\begin{Vector}  0 \\  0 \\  1 \end{Vector}
\\
  \mathrm A_{\Omega_0}^4&=\sqrt2\begin{Matrix3}
     0 &  0 & -1 \\
     0 &  0 &  1 \\
     1 & -1 &  0
  \end{Matrix3} & 
  \mathbf a_{\Omega_0}^4&=\sqrt2\begin{Vector}  0 \\  0 \\  0 \end{Vector}
\\
  \mathrm A_{\Omega_0}^5&=\psqrt\begin{Matrix3}
     0 &  0 &  0 \\
     0 &  0 & -1 \\
     0 &  1 &  0
  \end{Matrix3} & 
  \mathbf a_{\Omega_0}^5&=\psqrt\begin{Vector}  0 \\  0 \\  0 \end{Vector}
\end{align}
with the resulting shape functions
\begin{align}
  \psi_{\Omega_0}^0(\mathbf x)
    &=\psqrt\begin{pmatrix} 1-x_1 \\ x_0-1 \\ 0 \end{pmatrix} &
  \nabla\times\psi_{\Omega_0}^0
    &=\psqrt\begin{Vector}  0 \\  0 \\  2 \end{Vector}
\\
  \psi_{\Omega_0}^1(\mathbf x)
    &=\sqrt2\begin{pmatrix} x_1-x_2 \\ 1-x_0 \\ x_0-1 \end{pmatrix} &
  \nabla\times\psi_{\Omega_0}^1
    &=\sqrt2\begin{Vector}  0 \\ -2 \\ -2 \end{Vector}
\\
  \psi_{\Omega_0}^2(\mathbf x)
    &=\psqrt\begin{pmatrix} x_2-x_1 \\ x_0-x_2 \\ x_1-x_0 \end{pmatrix} &
  \nabla\times\psi_{\Omega_0}^2
    &=\psqrt\begin{Vector}  2 \\  2 \\  2 \end{Vector}
\\
  \psi_{\Omega_0}^3(\mathbf x)
    &=\sqrt3\begin{pmatrix} x_2 \\ 0 \\ 1-x_0 \end{pmatrix} &
  \nabla\times\psi_{\Omega_0}^3
    &=\sqrt3\begin{Vector}  0 \\  2 \\  0 \end{Vector}
\\
  \psi_{\Omega_0}^4(\mathbf x)
    &=\sqrt2\begin{pmatrix} -x_2 \\ x_2 \\ x_0-x_1 \end{pmatrix} &
  \nabla\times\psi_{\Omega_0}^4
    &=\sqrt2\begin{Vector} -2 \\ -2 \\  0 \end{Vector}
\\
  \psi_{\Omega_0}^5(\mathbf x)
    &=\psqrt\begin{pmatrix} 0 \\ -x_2 \\ x_1 \end{pmatrix} &
  \nabla\times\psi_{\Omega_0}^5
    &=\psqrt\begin{Vector}  2 \\  0 \\  0 \end{Vector}
\end{align}
To integrate arbitrary second order functions over tetrahedron 0 we use
$\int_{\Omega_0}dV\cdots=\int_0^1dx_0\int_0^{x_0}dx_1\int_0^{x_1}dx_2\cdots$.
That yields (with the help of maxima) the following contributions
\begin{align}
  T_{ 0, 0;\Omega_0}=\int_{\Omega_0}dV(\psi^0_{\Omega_0})^2
  &=\frac1{15} &
  S_{ 0, 0;\Omega_0}=\int_{\Omega_0}dV(\nabla\times\psi^0_{\Omega_0})^2
  &=\frac23 \\
  T_{ 2, 2;\Omega_0}=\int_{\Omega_0}dV(\psi^1_{\Omega_0})^2
  &=\frac1{10} &
  S_{ 2, 2;\Omega_0}=\int_{\Omega_0}dV(\nabla\times\psi^1_{\Omega_0})^2
  &=\frac83 \\
  T_{ 7, 7;\Omega_0}=\int_{\Omega_0}dV(\psi^2_{\Omega_0})^2
  &=\frac1{12} &
  S_{ 7, 7;\Omega_0}=\int_{\Omega_0}dV(\nabla\times\psi^2_{\Omega_0})^2
  &=2 \\
  T_{ 6, 6;\Omega_0}=\int_{\Omega_0}dV(\psi^3_{\Omega_0})^2
  &=\frac1{10} &
  S_{ 6, 6;\Omega_0}=\int_{\Omega_0}dV(\nabla\times\psi^3_{\Omega_0})^2
  &=2 \\
  T_{ 9, 9;\Omega_0}=\int_{\Omega_0}dV(\psi^4_{\Omega_0})^2
  &=\frac1{10} &
  S_{ 9, 9;\Omega_0}=\int_{\Omega_0}dV(\nabla\times\psi^4_{\Omega_0})^2
  &=\frac83 \\
  T_{13,13;\Omega_0}=\int_{\Omega_0}dV(\psi^5_{\Omega_0})^2
  &=\frac1{15} &
  S_{13,13;\Omega_0}=\int_{\Omega_0}dV(\nabla\times\psi^5_{\Omega_0})^2
  &=\frac23
\end{align}

\section{The Right Hand Side}

The right hand side is
\begin{equation}
  b_i=\begin{cases}
    (2Tu^n-(T+(\Delta t)^2S)u^{n-1})_i&\text{for $i$ non-constrained} \\
    0                                &\text{otherwise},
  \end{cases}
\end{equation}
with
\begin{align}
  T_{ij}&=\int_\Omega\epsilon\psi_i\cdot\psi_jdV \\
  S_{ij}&=\int_\Omega\mu^{-1}(\nabla\times\psi_i)\cdot(\nabla\times\psi_j)dV.
\end{align}
As noted above, we assume $\epsilon=1$, and from here on also $\mu=1$.

\end{document}

%%% Local Variables: 
%%% mode: latex
%%% mode: TeX-PDF
%%% mode: auto-fill
%%% TeX-master: t
%%% mode: flyspell
%%% ispell-local-dictionary: "american"
%%% End: 
